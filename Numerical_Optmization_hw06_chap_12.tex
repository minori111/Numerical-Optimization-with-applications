\documentclass[11pt,a4paper]{article}
\usepackage{amsmath}
\usepackage{amsthm}
\usepackage{amssymb}
\usepackage[margin=2cm]{geometry}
%\usepackage{thmbox}
\usepackage{graphicx}
\usepackage[dvipsnames,usenames]{color}
\usepackage{url}
\usepackage{comment}
\usepackage{amsmath, amsthm, amssymb,enumerate}

%\usepackage{enumerate}
%\usepackage{titlesec}
%\usepackage{Rvector}
%\usepackage{mathabx}
\newcommand{\qrq}{\quad\Rightarrow\quad}
\newcommand{\qarq}{\quad&\Rightarrow\quad}
\newcommand{\alp}{\alpha}
\newcommand{\claim}{{\underline{\it Claim:}}~~}
\newcommand{\dbR}{\mathbb{R}}
\newcommand{\ndimr}{\mathbb{R}^n}
\newcommand{\vare}{\varepsilon}
\newcommand{\since}{\because\;}
\newcommand{\hence}{\therefore\;}
\newcommand{\en}{\par\noindent}
\newcommand{\fn}{\footnotesize}

\newcommand{\sect}[2]{#1~~{\mdseries\tiny(#2)}}

\renewcommand{\(}{\left(}
\renewcommand{\)}{\right)}

\let \ds=\displaystyle

\usepackage{xeCJK}
\setCJKmainfont[AutoFakeBold=5,AutoFakeSlant=.4]{標楷體}

%\usepackage{fancyhdr}
%\pagestyle{fancy}
%\renewcommand{\headrulewidth}{0pt}

\renewcommand{\thesection}{Lecture \arabic{section}}
\renewcommand{\thesubsection}{\Roman{subsection}}

\usepackage[T1]{fontenc}

%%%% F U N C T I O N %%%%%
\newcommand{\abs}[1]{\left|#1\right|}
\newcommand{\norm}[1]{\left\|#1\right\|}
\newcommand{\inn}[1]{\left<#1\right>}
\newcommand{\f}[1]{f\!\left(#1\right)}
\newcommand{\g}[1]{g\!\left(#1\right)}
\newcommand{\h}[1]{h\!\left(#1\right)}
\newcommand{\x}[1]{x\!\left(#1\right)}
\newcommand{\D}[1]{D\!\left(#1\right)}
\newcommand{\N}[1]{N\!\left(#1\right)}
\renewcommand{\P}[1]{P\!\left(#1\right)}
\newcommand{\R}[1]{R\!\left(#1\right)}
\newcommand{\V}[1]{V\!\left(#1\right)}
\newcommand{\function}[2]{#1\!\left(#2\right)}
\newcommand{\functions}[2]{\left(#1\right)\!\left(#2\right)}

\definecolor{light-gray}{gray}{0.95}
\newcommand{\textfil}[1]{\colorbox{light-gray}{\large\color{Red} #1}}


\renewcommand{\title}{Numerical Optimization with applications: Homework 06}
\renewcommand{\author}{104021601 林俊傑\\104021602 吳彥儒\\104021615 黃翊軒}
\renewcommand{\maketitle}{\begin{center}\textbf{\Large\title}\\[6pt] {\author}\\[6pt] {\color{Gray}\footnotesize December 7, 2016}\end{center}}
\newcommand{\blue}[1]{{\color{blue}#1}}


\renewcommand{\labelenumi}{(\alph{enumi})}

\newcommand{\Exercise}[2]{\textbf{Exercise #1.} \textit{#2}}
\newtheorem{exercise}{Exercise}

%\parskip=11pt

\begin{document}

  \maketitle

\begin{exercise}
  	 The following example from [268] with a single variable $x \in \mathbb{R}$ and a single equality constraint shows that strict local solutions are not necessarily isolated. Consider
  	 $$
  	 \min_x x^2 \quad \text{subject to } c(x) = 0, \text{where } c(x) = \begin{cases}
	  	 x^6\sin (1/x) &\text{if $x\ne 0$}\\
	  	 0 &\text{if $x = 0$}
  	 \end{cases} \qquad \text{(12.96)}
  	 $$
  	 \begin{enumerate}[(a)]
  	 	\item Show that the constraint function is twice continuously differentiable at all $x$ (including at $x = 0$) and that the feasible points are $x = 0$ and $x = 1/(k\pi)$ for all nonzero integers $k$.
  	 	\item Verify that each feasible point except $x = 0$ is an isolated local solution by showing that there is a neighborhood $\mathcal{N}$ around each such point within which it is the only feasible point.
  	 \end{enumerate}
  \end{exercise}
  \begin{proof}
  	\begin{enumerate}
  		\item We first show directly that constraint function is twice continuously differentiable at all $x$.
  		
  		If $x \ne 0$, then
  		\begin{align*}
	  		c(x) &= x^6 \sin(1/x)\\
	  		c'(x) &= 6x^5\sin(1/x) - x^4\cos(1/x)\\
	  		c''(x) &= (30x^4 - x^2)\sin(1/x) - 10x^3\cos(1/x)
  		\end{align*}
  		Hence, the constraint function is twice continuously differentiable at all $x$.
  		
  		Second, we show that the feasible points are $x = 0$ and $x = 1/(k\pi)$ for all nonzero integers $k$.
  		
  		If $x = 0$, then $c(x) = 0$ by definition.
  		If $x = 1/(k\pi)$, then $\sin(1/x) = 0$ and thus we have $c(x) = 0$.
  		\item If $x = 1/(k\pi)$ for some fixed nonzero integers $k$ and we choose $r = \left|\frac{1}{k\pi}-\frac{1}{(k+1\pi)}\right|$, then the open interval $\mathcal{N} = (x-r, x+r)$ contains only a feasible point, which is $x$ itself. 
  		
  		On the other hand, if $x = 0$, then for all $r>0$ there exists nonzero positive integers $k$ such that $x_k = 1/(k\pi)<r$. However, $x_k$ are feasible points in the neighborhood $(-r,r)$.
  		
  		Therefore, combining disscusion above, we have that each feasible point except $x = 0$ is an isolated local solution by showing that there is a neighborhood $\mathcal{N}$ around each such point within which it is the only feasible point.
  	\end{enumerate}
  \end{proof}
  \newpage
  
  \setcounter{exercise}{14}
  \begin{exercise}
Consider the following modification of (12.36), where $t$ is a parameter to be fixed prior to solving the problem:
\begin{align*}
\min_{x} (x_{1}-\frac{3}{2})^{2}+(x_{2}-t)^{4}~~~~ s.t.~~~~~\begin{bmatrix}
1-x_{1}-x_{2}\\
1-x_{1}+x_{2}\\
1+x_{1}-x_{2}\\
1+x_{1}+x_{2}
\end{bmatrix}\geq0
\end{align*}
(a)For what value of $t$ does the point $x^{*}=(1,0)^{T}$ satisfy the KKT conditions?\\
(b)Show that when $t=1$, only the first constraint is active at the solution, and find the solution.
  \end{exercise}  
  \begin{proof}
(a) First, we check the complementary condition of KKT, i.e. $\lambda_{i}c_{i}(x^{*})=0$ for $i=1,2,3,4.$

\begin{equation}
\begin{cases}
\lambda_{1}(1-1-0)=0\\
\lambda_{2}(1-1+0)=0\\
\lambda_{3}(1+1-0)=0\\
\lambda_{4}(1+1+0)=0
\end{cases}
\Rightarrow
\begin{cases}
\lambda_{3}=0\\
\lambda_{4}=0
\end{cases}
\end{equation}
Obviously, $c(x^{*})\geq0$ holds. Consider 
\begin{align*}
\bigtriangledown_{x}L(x^{*},\lambda)&=\bigtriangledown_{x}L((0,1)^{T},\lambda)\\
&=\begin{bmatrix}
-1+\lambda_{1}+\lambda_{2}\\
-4t^{3}+\lambda_{1}-\lambda_{2}
\end{bmatrix}
=\begin{bmatrix}
0\\0
\end{bmatrix}
\end{align*} 
Since $\lambda_{1}, \lambda_{2}\geq0$ and $\lambda_{1}+\lambda_{2}=1$, we have $\lambda_{1}-\lambda_{2}\in[-1,1]$. Hence, $4t^{3}\in[-1,1]$, then $t\in[-\sqrt[3]{4},\sqrt[3]{4}]$.\\
(b) We know the feasible set $E=\lbrace x\in \mathbb{R}^{2}~|~ \Vert x\Vert_{1}=1\rbrace$. And compute the gradient of $f(x)$
\begin{align*}
\bigtriangledown f(x) &= 
\begin{bmatrix}
2(x_{1}-\frac{3}{2})\\
4(x_{2}-1)^{3}
\end{bmatrix}
\leq 0~~~~\forall x\in E\\
\end{align*}
From above, $\forall x,y \in E$, if $x_{1}\leq y_{1}$ and $x_{2}\leq y_{2}$ then $f(y)\leq f(x)$. Therefore, we only have to  consider the case that the first constraint is active: $1-x_{1}-x_{2}=0$. Substituting $x_{2}=1-x_{1}$ into $f(x)$ and find the minimum of $f$:
\begin{align*}
f(x)&=(x_{1}-\frac{3}{2})^{2}+(-x_{1})^{4}\\
 f'(x)&=2(x_{1}-\frac{3}{2})+4x_{1}^{3}\\
&=4x_{1}^{3}+2x_{1}-3
\end{align*}
$f'(x^{*})=0$ if $x^{*}_{1}=\frac{\sqrt[3]{27+\sqrt{753}}}{2\times 3^{2/3}}-\frac{1}{\sqrt[3]{3(27+\sqrt{753})}}$.
Consequently, the local minimum of $f$ is $(x^{*}_{1},1-x^{*}_{1})$. On the other hand, $E$ is a convex set and $f$ is convex on $E$. By exercise 12.4, $x^{*}$ is also a global minimum of $f$ on $E$.
  \end{proof}
  \newpage

  \setcounter{exercise}{18}
  \begin{exercise}
  Consider the problem
  \[ \min_{x \in R^2}=-2x_1+x_2 \quad \mbox{subject to} \quad 
  \begin{cases}
  (1-x_1)^3-x_2, & \geq 0 \\
  x_2+0.25x_1^2-1,& \geq 0
  \end{cases} \]
  The optimal solution is $x^\ast=(0,1)^T$, where both constraints are active.
  \begin{enumerate}[(a)]
  \item Do the LICQ hold at this point?
  \item Are the KKT conditions satisfied?
  \item Write down the set $\mathcal{F}(x^\ast)$ and $\mathcal{C}(x^\ast,\lambda^\ast)$.
  \item Are the second-order necessary conditions satisfied? Are the second-order sufficient conditions satisfied?
  \end{enumerate}
  \end{exercise}  
  \begin{proof}
  \begin{enumerate}[(a)]
  \item \[ A(x^\ast)=\left[ \bigtriangledown C_i(x^\ast) \right]_{i \in \mathcal{A}(x^\ast)} =
  \begin{bmatrix}
  -3(1-x_1)^2 & 0.5x_1\\
  -1 & 1
  \end{bmatrix}_{x=x\ast}=
  \begin{bmatrix}
  -3 & 0\\
  -1 & 1
  \end{bmatrix}\]
  $A(x^\ast)$ is nonsingular. Therefore, the LICQ holds.
  \item \[ \bigtriangledown f(x^\ast)=
  \begin{bmatrix}
  -2\\
  1
  \end{bmatrix}, \quad
  \bigtriangledown c_1(x^\ast)=
  \begin{bmatrix}
  -3\\
  -1
  \end{bmatrix}, \quad
  \bigtriangledown c_2(x^\ast)=
  \begin{bmatrix}
  0\\
  1
  \end{bmatrix}\]
  Therefore, the KKT conditions (12.34a)-(12.34e) are satisfied when we set
  \[ \lambda^\ast= \left(\frac{2}{3},\frac{5}{3}\right)^T\]
  \item \[ \mathcal{F}(x^\ast)=\lbrace d\vert \bigtriangledown c_i(x^\ast)^Td \geq 0 \rbrace =\lbrace (d_1,d_2)^T \vert 
  \begin{bmatrix}
  -3 & -1\\
  0 & 1
  \end{bmatrix}
  \begin{bmatrix}
  d_1\\
  d_2  
  \end{bmatrix}     \geq 0 \rbrace
  =\lbrace (d_1,d_2)^T \vert d_2 \geq 0,~3d_1+d_2 \leq 0 \rbrace \]
  \begin{align*}
  \mathcal{C}(x^\ast,\lambda^\ast) 
  &=\lbrace w \in \mathcal{F}(x^\ast) \vert \bigtriangledown c_i(x^\ast)^Tw=0~\forall i \in \mathcal{A}(x^\ast)\cap I \mbox{ with } \lambda_i^\ast > 0 \rbrace\\
  &=\lbrace w \in \mathcal{F}(x^\ast) \vert \bigtriangledown c_i(x^\ast)^Tw=0~\mbox{for }i=1,2 \rbrace\\
  &=\lbrace (w_1,w_2)^T \in \mathcal{F}(x^\ast) \vert
  \begin{bmatrix}
  -3 & -1\\
  0 & 1
  \end{bmatrix}
  \begin{bmatrix}
  w_1\\
  w_2  
  \end{bmatrix}=0 \rbrace\\
  &=\lbrace (0,0)^T \rbrace
  \end{align*}
  \item \begin{equation}\label{E:condi}
  \forall w \in \mathcal{C}(x^\ast,\lambda^\ast)={(0,0)^T} \qquad w^T \bigtriangledown_x \mathcal{L}(x^\ast,\lambda^\ast)w=0
  \end{equation}
  Since at $x^\ast$ LICQ holds and $(x^\ast,\lambda^\ast)$ satisfies KKT. By (\ref{E:condi}), the necessary condition is satisfied.\\
  Since  $x^\ast$ is a feasible solution and $(x^\ast,\lambda^\ast)$ satisfies KKT. By (\ref{E:condi}), the sufficient condition is satisfied.
  
  \end{enumerate}
  \end{proof}
\end{document} 